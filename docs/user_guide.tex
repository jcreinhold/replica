\documentclass[12pt]{article}

%%%%%%%%%%%%%% PACKAGES %%%%%%%%%%%%%%%

% Formatting
\usepackage[utf8]{inputenc}
\usepackage[margin=1in]{geometry}
\usepackage{hyperref}
\usepackage{parskip}

%%%%%%%%%%%% INFORMATION %%%%%%%%%%%%%%%

\title{\texttt{replica} Documentation}
\author{Jacob Reinhold}

%%%%%%%%%%%%%% DOCUMENT %%%%%%%%%%%%%%%%

\begin{document}

\maketitle

\section{Overview}

This package implements the REPLICA image synthesis package outlined in Jog, et
al. 2017 \cite{jog} in MATLAB. This package has been superseded by the
\href{https://github.com/jcreinhold/synthit}{synthit} package. Note that this
package is \emph{not} actively maintained and only put up for archival purposes.

To use the package, set your MATLAB path to include the \texttt{src} directory
and all of its subfolders. This will allow the package to reach all required 
functions.

An initial step for all processing will be to create a parameter structure,
dubbed a \texttt{param\_struct} in the codebase. A default/template
\texttt{param\_struct} can be constructed by the functions in
\texttt{src/utilties/singleres/default\_param\_struct.m} and 
\texttt{src/utilties/multires/default\_param\_struct\_multires.m}, depending on
whether or not the user wants to synthesize skull-stripped images or not
(respectively).

\section{Training}

For skull-stripped images, you can use \texttt{replica\_train.m} and for
non-skull-stripped images you should use \texttt{replica\_train\_multires.m}. If
you are running this on a memory-constrained system, you should use
\texttt{replica\_train\_multires\_low\_memory.m} (note that this takes longer
than the non-low-memory version).

\section{Prediction}

Use the corresponding \texttt{replica\_predict\_*.m} according to what you used
for training.

\section{Miscellaneous}

If you have difficulty using this package, you can use the original version
listed \href{https://www.nitrc.org/frs/?group_id=834}{here}. Note that the
multi-resolution (non-skull-stripped) version is not implemented in this link.

\begin{thebibliography}{9}
\bibitem{jog} A. Jog, A. Carass, S. Roy, D. L. Pham, and J. L. Prince, 
``Random forest regression for magnetic resonance image synthesis,'' 
Med. Image Anal., vol. 35, pp. 475–488, 2017.
\end{thebibliography}


\end{document}
